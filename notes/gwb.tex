\documentclass[11pt]{article}
\usepackage{latexsym}
\usepackage{graphicx}
\usepackage{amssymb,amsmath,amsthm,longtable}
\usepackage{epsfig}
\usepackage{epsf}
\usepackage[top=1in, bottom=1in, left=1in, right=1in]{geometry}

% more packages
%\usepackage{draftcopy}
%\usepackage{aas_macros}
\numberwithin{equation}{section}
\usepackage{hyperref}

% draft watermark
\usepackage{draftwatermark}
\SetWatermarkText{DRAFT}
\SetWatermarkColor[gray]{0.9}
\SetWatermarkScale{7}

% begin equation, itemize, etc.
\def\be{\begin{equation}}
\def\ee{\end{equation}}
\def\bi{\begin{itemize}}
\def\ei{\end{itemize}}
\def\ben{\begin{enumerate}}
\def\een{\end{enumerate}}
\def\i{\item{}}
\def\edth{\check\partial}

% special math fonts
\newcommand{\bs}[1]{\boldsymbol{#1}}
\newcommand{\mb}[1]{\mathbf{#1}}
\newcommand{\mbb}[1]{\mathbb{#1}}
\newcommand{\mc}[1]{\mathcal{#1}}
\newcommand{\mf}[1]{\mathfrak{#1}}
\newcommand{\mr}[1]{\mathrm{#1}}
\newcommand{\ms}[1]{\mathsf{#1}}
\newcommand{\mat}[1]{\mathsf{#1}}
\newcommand{\unit}{\mathds{1}}
\newcommand{\zero}{\mb{0}}
 
% some more definitions
\def\cross{\times}
\def\del{\nabla}
\def\grad{\vec\nabla}
\def\div{\grad\cdot}
\def\curl{\grad\cross}
\def\I{I\!\!\!-}
 
\def\defn{\underline{Definition}:\ }
\def\ex{\underline{Example}:\ }
\def\exer{\underline{Exercise}:\ }
\def\soln{\underline{Solution}:\ }
\def\theor{\underline{Theorem}:\ }
\def\pf{\underline{Proof}:\ }
\def\ques{\underline{Question}:\ }
\def\ans{\underline{Answer}:\ }
 
%%%%%%%%%%%%%%%%%%%%%%%%%%%%%%%%%%%%%%%%%%%%%%%%%%%%%%%%%%%%%%
\begin{document}

\title{Searches for stochastic gravitational-wave backgrounds}
\author{Joseph D.\ Romano}
\date{Les Houches Summer School\\
July 2018}
\maketitle

\abstract{These lecture notes provide a brief introduction 
to detection methods used to search for a stochastic 
background of gravitational radiation---a superposition
of gravitational-wave signals either too weak or too 
numerous to individually detect.
The lectures are divided into two main pieces:
(i) an overview, with a description of different types 
of stochastic backgrounds and an introduction to the
correlation method using multiple detectors;
(ii) details, extending the previous discussion to 
non-trivial detector response, what to do in the absence
of correlations, and a recently proposed Bayesian method 
to search for the gravitational-wave background produced 
by stellar-mass binary black hole mergers throughout the 
universe.

\tableofcontents
\pagebreak

%%%%%%%%%%%%%%%%%%%%%%%%%%%%%%%%%%%%%%%%%%%%%%%%%%
% noindent, space between paragraphs
\setlength{\parindent}{0pt}
\setlength{\parskip}{\medskipamount}
%%%%%%%%%%%%%%%%%%%%%%%%%%%%%%%%%%%%%%%%%%%%%%%%%%

\section{Motivation}

A stochastic background of gravitational radiation 
is a superposition of gravitational-wave signals either 
too weak or too numerous to individually detect.
The individual signals making up the background are thus
{\em unresolvable}, unlike the large signal-to-noise 
binary black-hole (BBH) and binary neutron-star (BNS)
merger signals recently detected by the advanced LIGO 
and Virgo detectors.

Stochastic backgrounds can be of either astrophysical 
or cosmological origin:

(i) A potential astrophysical background for the 
current generation of ground-based interferometers 
is the combined signal from the population of 
stellar-mass BBH and BNS mergers throughout the universe.
We will discuss the prospects of detecting this 
potential background throughout these lectures; the 
last section is devoted to a recently proposed 
detection method that targets this particular source.

(ii) A potential cosmological background is formed 
from {\em relic gravitational waves}---that is, 
quantum fluctuations in the geometry of space-time,
driven to macroscopic scales 
by a period of rapid expansion (e.g., inflation) 
a mere $\sim 10^{-32}~{\rm s}$ after the Big Bang.
This relic background is too weak to be detected by 
advanced LIGO, Virgo, etc.,
but is potentially detectable by its effect on the 
polarization of the cosmic microwave background (CMB)
radiation.
The Planck satellite and BICEP experiment (located
at the South Pole) are searching for this signal.

The ultimate goal of gravitational-wave background
searches is to produce the GW analogue of Figure~\ref{f:CMB}.
Figure~\ref{f:CMB} is a sky map of the temperature 
fluctuations in the CMB blackbody radiation, relative 
to the $T_0 = 2.73~{\rm K}$ isotropic component.
(The dipole contribution due to our motion with respect 
to the cosmic rest frame has also been subtracted out.)
Recall that the CMB is a background of electromagnetic
radiation, produced at the time of last scattering,
roughly 380,000~yr after the Big Bang.
At that time, the universe had a temperature of 
$\sim 3000~{\rm K}$, roughly one thousand times larger 
than the temperature today, but cool enough for 
neutral hydrogen atoms to first form and photons to 
propagate freely.
The temperature fluctuations in the CMB radiation tell
us about the density of matter on the surface of last 
scattering and also about the integrated ...

For perspective, this map of the temperature fluctuations 
in the CMB was produced by the Planck mission in 2013,
almost 50 years after its initial detection by Penzias
and Wilson in 1965.

\section{Different types of stochastic backgrounds}

\section{Mathematical characterization of a stochastic background}


\section{Correlation methods - basic idea}

\section{Some simple examples}

We now apply the above correlation method

\section{Non-trivial detector response}

\section{Non-trivial correlations}

\section{What to do in the absence of correlations?}

\section{Searching for the background of binary black-hole
mergers}

\input refs

\end{document}

