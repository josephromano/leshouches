\documentclass[11pt]{article}
\usepackage{latexsym}
\usepackage{graphicx}
\usepackage{amssymb,amsmath,amsthm,longtable}
\usepackage{epsfig}
\usepackage{epsf}
\usepackage{caption}

% more packages
%\usepackage{draftcopy}
%\usepackage{aas_macros}
\usepackage{hyperref}

% draft watermark
\usepackage{draftwatermark}
\SetWatermarkText{DRAFT}
\SetWatermarkColor[gray]{0.9}
\SetWatermarkScale{7}

\captionsetup{width=5.75in}
\usepackage[top=1in, bottom=1in, left=1in, right=1in]{geometry}

\numberwithin{equation}{section}

% begin equation, itemize, etc.
\def\be{\begin{equation}}
\def\ee{\end{equation}}
\def\bi{\begin{itemize}}
\def\ei{\end{itemize}}
\def\ben{\begin{enumerate}}
\def\een{\end{enumerate}}
\def\i{\item{}}
\def\edth{\check\partial}

% special math fonts
\newcommand{\bs}[1]{\boldsymbol{#1}}
\newcommand{\mb}[1]{\mathbf{#1}}
\newcommand{\mbb}[1]{\mathbb{#1}}
\newcommand{\mc}[1]{\mathcal{#1}}
\newcommand{\mf}[1]{\mathfrak{#1}}
\newcommand{\mr}[1]{\mathrm{#1}}
\newcommand{\ms}[1]{\mathsf{#1}}
\newcommand{\mat}[1]{\mathsf{#1}}
\newcommand{\unit}{\mathsf{1}}
\newcommand{\zero}{\mathsf{0}}
 
% some more definitions
\def\cross{\times}
\def\del{\nabla}
\def\grad{\vec\nabla}
\def\div{\grad\cdot}
\def\curl{\grad\cross}
\def\I{I\!\!\!-}
\def\D{{\rm d}}
 
\def\defn{\underline{Definition}:\ }
\def\ex{\underline{Example}:\ }
\def\exer{\underline{Exercise}:\ }
\def\soln{\underline{Solution}:\ }
\def\theor{\underline{Theorem}:\ }
\def\pf{\underline{Proof}:\ }
\def\ques{\underline{Question}:\ }
\def\ans{\underline{Answer}:\ }
 
%%%%%%%%%%%%%%%%%%%%%%%%%%%%%%%%%%%%%%%%%%%%%%%%%%%%%%%%%%%%%%
\begin{document}

\title{Searches for stochastic gravitational-wave backgrounds}
\author{Joseph D.\ Romano}
\date{Les Houches Summer School\\
July 2018}

\maketitle

\abstract{These lecture notes provide a brief introduction 
to detection methods used to search for a stochastic 
background of gravitational radiation---i.e., a 
superposition of gravitational-wave signals either too weak 
or too numerous to individually detect.
The lectures are divided into two main pieces:
(i) an overview, consisting of a description of different 
types of gravitational-wave backgrounds and an introduction 
to the correlation method using multiple detectors;
(ii) details, extending the previous discussion to 
non-trivial detector response, what to do in the absence
of correlations, and a recently proposed Bayesian method 
to search for the gravitational-wave background produced 
by stellar-mass binary black hole mergers throughout the 
universe.
Suggested exercises for the reader are given throughout 
the text.}

\pagebreak
\tableofcontents
\pagebreak

%%%%%%%%%%%%%%%%%%%%%%%%%%%%%%%%%%%%%%%%%%%%%%%%%%
% noindent, space between paragraphs
%\setlength{\parindent}{0pt}
%\setlength{\parskip}{\medskipamount}
%%%%%%%%%%%%%%%%%%%%%%%%%%%%%%%%%%%%%%%%%%%%%%%%%%

\section{Motivation}

A stochastic background of gravitational radiation 
is a superposition of gravitational-wave signals either 
too weak or too numerous to individually detect.
The individual signals making up the background are thus
{\em unresolvable}, unlike the large signal-to-noise 
binary black-hole (BBH) and binary neutron-star (BNS)
merger signals recently detected by the advanced LIGO 
and Virgo detectors.
But despite the fact that the individual signals are 
unresolvable, the detection of a stochastic 
gravitational-wave background (GWB) will 
be able to provide information about the 
{\em statistical} properties of the source.

\subsection{Gravitational-wave analogue of the CMB}

The ultimate goal of gravitational-wave background
searches is to produce the GW analogue of Figure~\ref{f:CMB}.
%
\begin{figure}[htbp!]
\begin{center}
\includegraphics[width=0.6\textwidth]{Figures/CMB}
\caption{Skymap of $\Delta T/T_0$ for the cosmic microwave background
radiation produced by the Planck 2013 mission.}
\label{f:CMB}
\end{center}
\end{figure}
%
This is a sky map of the temperature fluctuations in 
the CMB blackbody radiation, $\Delta T/T$, relative 
to the $T_0 = 2.73~{\rm K}$ isotropic component.
(The dipole contribution due to our motion with respect 
to the cosmic rest frame has also been subtracted out.)
Recall that the CMB is a background of electromagnetic
radiation, produced at the time of last scattering,
roughly 380,000~yr after the Big Bang.
At that time, the universe had a temperature of 
$\sim 3000~{\rm K}$, approximately one thousand times 
larger than the temperature today, but cool enough for 
neutral hydrogen atoms to first form and photons to 
propagate freely.
The temperature fluctuations in the CMB radiation tell
us about the density of matter on the surface of last 
scattering, thus giving us a picture of the ``seeds" of 
large-scale structure formation in the early universe.

For perspective, Figure~\ref{f:CMB} was produced by 
the Planck mission in 2013,
almost 50 years after the CMB radiation was initially
detected by Penzias and Wilson in 1965.
It took many years and improved experiments
(COBE, Boomerang, WMAP, Planck to name a few) to get to 
the high-precision measurements that we have today.
So its somewhat sobering to realize that right now, 
in 2018, we have yet to detect the isotropic component 
of the GWB.
So we have a long road ahead of us.  

\subsection{The background of BBH and BNS mergers in the 
LIGO band}

But fortunately, as mentioned above, the advanced LIGO
and Virgo detectors have detected other 
gravitational-wave signals from several individual BBH 
and BNS mergers.
These were very strong signals, having 
matched-filter signal-to-noise ratios ${\rm SNR}\gtrsim 10$, 
and false alarm probabilities $<2\times 10^{-7}$,
corresponding to 5-sigma ``gold-plated" events.
Similar large SNR detections are expected during the 
upcoming observing run O3, which is scheduled to start in early 2019.
But we also expect that there are many more signals, 
corresponding to more distant mergers or smaller mass systems, 
which are 
individually undetectable (i.e., {\em subthreshold} events).
This weaker background of gravitational radiation is 
nonetheless detectable {\em as a collectivity} via the common 
influence of the gravitational waves on multiple detectors.

To get an idea of the statistical properties of this
background signal, we can estimate the total rate of 
stellar-mass BBH mergers throughout the universe by
using the local rate estimate from these first detections,
$9$-$240~{\rm Gpc}^{-3}\,{\rm yr}^{-1}$.
This leads to a prediction for the total rate of 
BBH mergers between $\sim\!1$~per minute to a few per hour.
(\exer 1: Verify these values for the total rate.%
\footnote{A more complete description of this and 
all other exercises is given in Section~\ref{s:exercises}.})
Since the duration of BBH merger signals in band 
is $\sim\!1~{\rm s}$, which is much smaller than the 
average duration between successive mergers, the 
combined signal will be highly-nonstationary (or {\em popcorn}-like).
We can perform similar calculations for BNS mergers.
The predicted total rate for such events is roughly 
one event every $15~{\rm s}$, while the duration of 
a BNS signal in band is roughly 100~{\rm s}. 
Thus, the BNS merger signals overlap in time leading to 
a continuous (or {\em confusion-limited}) background.
Figure~\ref{f:BBH-BNS-timeseries} is a plot of the
expected time-domain signal corresponding the rate
estimates calculated above.
%
\begin{figure}[htbp!]
\begin{center}
\includegraphics[width=0.6\textwidth]{Figures/BBH-BNS-timeseries}
\caption{Simulated time-domain signal for the predicted BBH and 
BNS background.  
Figure taken from \cite{stoch-implications}.}
\label{f:BBH-BNS-timeseries}
\end{center}
\end{figure}
%

The combined signal from BBH and BNS mergers is 
potentially detectable with advanced LIGO and Virgo, 
shortly after reaching design sensitivity.
Although the signal-to-noise ratios for the 
individual events are small, the combined 
signal-to-noise ratio of the correlated data 
summed over all events grows like the square-root 
of the observation time, reaching a detectable
level of $3$-sigma after roughly 40~months of
observation (Figure~\ref{f:BBH-BNS-SNR}).
%
\begin{figure}[htbp!]
\begin{center}
\includegraphics[width=0.6\textwidth]{Figures/BBH-BNS-SNR}
\caption{Expected signal-to-noise ratio of the correlated 
data for the advanced LIGO and Virgo detectors as a function 
of observation time.
The points labeled O1, O2, etc., indicate the start of
advanced LIGO's first observation run, second observation
run, etc. 
Figure taken from \cite{stoch-implications}.}
\label{f:BBH-BNS-SNR}
\end{center}
\end{figure}
%
This estimate of time to detection is based on 
the standard cross-correlation search (Section~\ref{s:correlations}), 
which assumes a Gaussian-stationary background.
But there is a better method, recently proposed by 
Smith and Thrane~\cite{Smith-Thrane:2018}, 
which should reduce the time to detection by several 
orders of magnitude (factor of $\sim\!1000$), 
meaning that the background would be detectable after only
a few days of operation. 
We will describe this method in more detail in 
Section~\ref{s:nonstationary}.

%%%%%%%%%%%%%%%%%%%%%%%%%%%%%%%%%%%%%%%%%%%%%%%%%%%%%%%
\section{Different types of stochastic backgrounds}
\label{s:different_types}

The combined signal from stellar-mass BBH and BNS 
mergers is just one way of producing a GWB, relevant 
for the current generation of ground-based laser 
interferometers like LIGO and Virgo.
Heavier-mass systems, which produce lower-frequency 
gravitational waves, are also expected to give rise 
to GWBs that are potentially detectable with other 
existing or proposed detectors.
Figure~\ref{f:GWspectrum} is a plot of the 
gravitational-wave spectrum, with frequencies ranging
from $10^3~{\rm Hz}$ (for ground-based interferometers)
to $10^{-17}~{\rm Hz}$ (corresponding to a period
equal to the age of the universe), together with 
potential sources and relevant detectors.  
Of particular note is the combined gravitational-wave
signal produced by compact white-dwarf binaries in 
the Milky Way, producing a ``confusion-limited" GWB
in the frequency band $\sim 10^{-4}~{\rm Hz}$ to 
$10^{-1}~{\rm Hz}$ relevant for the proposed space-based
interferometer LISA (expected launch date 2034).
This signal is expected to be so strong that it will
larger than the instrumental noise, forming actual 
a gravitational-wave ``foreground", rather than background,
signal.
almost 20 orders of magnitude (from kHz
\begin{figure}[htbp!]
\begin{center}
\includegraphics[width=0.9\textwidth]{Figures/GWspectrum}
\caption{Detectors and potential sources of gravitational-wave
backgrounds across the gravitational-wave spectrum.}
\label{f:GWspectrum}
\end{center}
\end{figure}
%
Stochastic backgrounds can be of either astrophysical 
or cosmological origin:

(i) A potential astrophysical background for the 
current generation of ground-based interferometers 
is the combined signal from the population of 
stellar-mass BBH and BNS mergers throughout the universe.
We will discuss the prospects of detecting this 
potential background throughout these lectures; the 
last section is devoted to a recently proposed 
detection method that targets this particular source.

(ii) A potential cosmological background is formed 
from {\em relic gravitational waves}---that is, 
quantum fluctuations in the geometry of space-time,
driven to macroscopic scales 
by a period of rapid expansion (e.g., inflation) 
a mere $\sim 10^{-32}~{\rm s}$ after the Big Bang.
This relic background is too weak to be detected by 
advanced LIGO, Virgo, etc.,
but is potentially detectable by its effect on the 
polarization of the cosmic microwave background (CMB)
radiation.
The Planck satellite and BICEP experiment (located
at the South Pole) are searching for this signal.

%%%%%%%%%%%%%%%%%%%%%%%%%%%%%%%%%%%%%%%%%%%%%%%%%%%%%%%
\section{Mathematical characterization of a stochastic background}

%%%%%%%%%%%%%%%%%%%%%%%%%%%%%%%%%%%%%%%%%%%%%%%%%%%%%%%
\section{Correlation methods - basic idea}
\label{s:correlations}

%%%%%%%%%%%%%%%%%%%%%%%%%%%%%%%%%%%%%%%%%%%%%%%%%%%%%%%
\section{Some simple examples}
\label{s:simple_examples}

We now apply the above correlation method

%%%%%%%%%%%%%%%%%%%%%%%%%%%%%%%%%%%%%%%%%%%%%%%%%%%%%%%
\section{Non-trivial detector response}

%%%%%%%%%%%%%%%%%%%%%%%%%%%%%%%%%%%%%%%%%%%%%%%%%%%%%%%
\section{Non-trivial correlations}

%%%%%%%%%%%%%%%%%%%%%%%%%%%%%%%%%%%%%%%%%%%%%%%%%%%%%%%
\section{What to do in the absence of correlations?}

%%%%%%%%%%%%%%%%%%%%%%%%%%%%%%%%%%%%%%%%%%%%%%%%%%%%%%%
\section{Searching for the background of binary black-hole
mergers}
\label{s:nonstationary}

%%%%%%%%%%%%%%%%%%%%%%%%%%%%%%%%%%%%%%%%%%%%%%%%%%%%%%
\newpage
\section{Exercises}
\label{s:exercises}

A more detailed description of the suggested exercises.

\ben

\i {\em Rate estimate of stellar-mass binary black hole mergers:}

Estimate the total rate (number of events per time) of 
stellar-mass binary black hole mergers throughout the universe 
by multiplying LIGO's O1 local rate 
estimate $R_0 \sim 10$~-~$200~{\rm Gpc}^{-3}\,{\rm yr}^{-1}$ by 
the comoving volume out to some large redshift, e.g., $z= 10$.
(For this calculation you can ignore any dependence of the 
rate density with redshift.)
You should find a merger rate of $\sim\!1$~per minute to a few 
per hour.

{\em Hint}: You will need to do numerically evaluate the
following integral for proper distance today as a function 
of source redshift:
%
\be
d_0(z) = \frac{c}{H_0}\int_0^z\frac{\D z'}{E(z')}\,,
\qquad
E(z)\equiv \sqrt{\Omega_{\rm m}(1+z)^3 + \Omega_\Lambda}\,,
\ee
%
with 
%
\be
\Omega_{\rm m}=0.31\,,
\qquad
\Omega_\Lambda=0.69\,,
\qquad
H_0 = 68~{\rm km}\,{\rm s}^{-1}\, {\rm Mpc}^{-1}\,.
\ee
Doing that integral, you should find what's shown in
Figure~\ref{f:d0vsz}, which you can then evaluate at
$z=10$ to convert $R_0$ (number of events per 
comoving volume per time) to total rate (number of 
events per time) for sources out to redshift $z=10$.
%
\begin{figure}[htbp!]
\begin{center}
\includegraphics[width=0.5\textwidth]{Figures/d0vsz}
\caption{}
\label{f:d0vsz}
\end{center}
\end{figure}
%

\i {\em Relating $S_h(f)$ and $\Omega_{\rm gw}(f)$:}

Derive the relationship 
\be
S_h(f) = \frac{3 H_0^2}{2\pi^2}\frac{\Omega_{\rm gw}(f)}{f^3}
\ee
between the strain power spectral density $S_h(f)$ and the 
dimensionless fractional energy density spectrum $\Omega_{\rm gw}(f)$.
({\em Hint}: You will need to use the various definitions of these
quantities and also 
\be
\rho_{\rm gw} =\frac{c^2}{32\pi G}\langle \dot h_{ab}(t,\vec x)\dot h^{ab}(t,\vec x)\rangle\,,
\ee
which expresses the energy-density in gravitational-waves to 
the metric perturbations $h_{ab}(t,\vec x)$.)

\i {\em Cosmology and the ``Phinney formula" for astrophysical backgrounds:}

(a) Using the Friedmann equation
%
\be
\left(\frac{\dot a}{a}\right)^2
=H_0^2\left(\frac{\Omega_{\rm m}}{a^{3}} + \Omega_\Lambda\right)
\ee
%
for a spatially-flat FRW spacetime with matter and 
cosmological constant, and the relationship 
%
\be
1+z = \frac{1}{a(t)}\,,
\qquad a(t_0)\equiv 1\quad(t_0\equiv {\rm today})\,,
\ee
%
between redshift $z$ and scale factor $a(t)$,
derive 
%
\be
\frac{\D t}{\D z} =-\frac{1}{(1+z)H_0 E(z)}\,,
\qquad
E(z) = \sqrt{\Omega_{\rm m}(1+z)^3 + \Omega_\Lambda}\,.
\ee
%
(b) Using this result for $\D t/\D z$, show that 
%
\be
\Omega_{\rm gw}(f)= \frac{f}{\rho_{\rm c}H_0}
\int_0^\infty \D z\>R(z)\,\frac{1}{(1+z)E(z)}
\left(\frac{\D E_{\rm gw}}{\D f_{\rm s}}\right)\bigg|_{f_{\rm s}=f(1+z)}
\ee
%
in terms of the rate density $R(z)$ as measured in 
the source frame 
(number of events per comoving volume per time interval
in the source frame).
({\em Hint}: The expression for $\D t/\D z$ from part
(a) will allow 
you to go from the ``Phinney formula" for
$\Omega_{\rm gw}(f)$ written in terms of the number 
density $n(z)$,
%
\be
\Omega_{\rm gw}(f)= \frac{1}{\rho_c}\int_0^\infty \D z\>
n(z)\,\frac{1}{1+z}\left(f_{\rm s}\,
\frac{\D E_{\rm gw}}{\D f_{\rm s}}\right)\bigg|_{f_{\rm s}=f(1+z)}\,,
\ee
%
to one in terms of the rate density 
$R(z)$, where $n(z)\,\D z=R(z)\,|\D t|_{t=t(z)}$.
Note: Both of the above expressions for $\Omega_{\rm gw}(f)$
assume that there is only one type of source, described by 
some set of average source parameters.  
If there is more than one type of source, one must sum
the contributions of each source to $\Omega_{\rm gw}(f)$.)

\i {\em Optimal filtering for the cross-correlation statistic:}

Verify the form 
%
\be
\tilde Q(f)\propto \frac{\Gamma_{12}(f)H(f)}
{P_1(f)P_2(f)}\,,
\ee
of the optimal filter function in the weak-signal limit,
where $H(f)$ is the assumed spectral shape of the 
gravitational-wave background,
$\Gamma_{12}(f)$ is the overlap function, and $P_1(f)$, $P_2(f)$ 
are the power spectral densities of the outputs of the 
two detectors (which are approximately equal to 
$P_{n_1}(f)$, $P_{n_2}(f)$, respectively).
Recall that the optimal filter $\tilde Q(f)$ maximizes
the signal-to-noise ratio of the cross-correlation 
statistic.
({\em Hint}: Introduce an inner product on the space of
functions of frequency $A(f)$, $B(f)$:
%
\be
(A,B)\equiv\int df A(f) B^*(f) P_1(f) P_2(f)\,.
\ee
%
This inner product
has all of the properties of the familiar dot product
of vectors in 3-dimensional space.
The signal-to-noise ratio of the cross-correlation
statistic can be written in terms of this inner product.)

\i {\em Maximum-likelihood estimators for single and multiple
parameters:}

(a) Show that the maximum-likelihood estimator $\hat a$ of 
the single parameter $a$ in the likelihood function
\be
p(d|a, \sigma) \propto
\exp\left[-\frac{1}{2}\sum_{i=1}^N \frac{(d_i-a)^2}{\sigma_i^2}\right]
\ee
%
is given by the noise-weighted average
%
\be
\hat a={\sum_i \frac{d_i}{\sigma_i^2}}\bigg/{\sum_j \frac{1}{\sigma_j^2}}\,.
\ee
%
(b) Extend the previous calculation to the likelihood
\be
p(d|A, C) \propto
\exp\left[-\frac{1}{2}(d-MA)^\dagger C^{-1} (d-MA)\right]\,,
\ee
%
where $A\equiv A_\alpha$ is a vector of parameters,
$C\equiv C_{ij}$ is the noise covariance matrix, and 
$M\equiv M_{i\alpha}$ is the response matrix mapping 
$A_\alpha$ to data samples, $MA\equiv \sum_\alpha M_{i\alpha}A_\alpha$.
For this more general case you should find:
%
\be
\hat A = F^{-1} X\,,
\ee
%
where
%
\be
F \equiv M^\dagger C^{-1} M\,,\qquad
X \equiv M^\dagger C^{-1} d\,.
\ee
%
In general, the matrix $F$ (called the {\em Fisher} matrix)
is not invertible, so some sort of regularization is needed
to do the matrix inversion.

\i {\em Timing-residual response for a 1-arm, 1-way detector:}

Derive the timing residual reponse function
%
\be
R^A(f,\hat n) = 
\frac{1}{2}u^a u^b e^A_{ab}(\hat n)
\frac{1}{i2\pi f}
\frac{1}{1+\hat n\cdot \hat u}
\left[1-e^{-\frac{i2\pi fL}{c}(1+\hat n\cdot\hat u)}\right]
\ee
%
for a single-link (i.e., a one-arm, one-way detector like 
that for pulsar timing).
Here $\hat u$ is the direction of propagation of the
electromagnetic pulse, and $\hat n$ is the direction to the 
GW source (the direction of wave propagation is 
$\hat k\equiv -\hat n$ and the direction to the pulsar is
$\hat p\equiv -\hat u$).
The origin of coordinates is taken to be at the position 
of the detector.

\i {\em Overlap function for colocated electric dipole antennae:}

Show that the overlap function for a pair of (short)
colocated electric dipole antennae pointing in directions 
$\hat u_1$ and $\hat u_2$ is given by 
%
\be
\Gamma_{12} 
%\equiv \langle r_1 r_2\rangle
\propto
\hat u_1\cdot\hat u_2 
\equiv\cos\zeta
\ee
% 
for the case of an unpolarized, isotropic electromagnetic field.
({\em Hint}: ``short" means that the phase of the electric 
field can be taken to be constant over of the lengths of 
the dipole antennae, 
so that the reponse of antenna $I=1,2$ to the field is
given by $r_I(t)=\hat u_I\cdot\vec E(t, \vec x_0)$, where
$\vec x_0$ is the common location of the two antenna.)
 
\i {\em Maximum-likelihood estimators for the standard 
cross-correlation statistic:}

\label{exer:MLestimators} 
Verify that 
%
\be
\hat C_{11}\equiv \frac{1}{N}\sum_{i=1}^N d_{1i}^2\,,
\qquad
\hat C_{22}\equiv \frac{1}{N}\sum_{i=1}^N d_{2i}^2\,,
\qquad
\hat C_{12}\equiv \frac{1}{N}\sum_{i=1}^N d_{1i} d_{2i}
\ee
%
are maximum-likelihood estimators of 
%
\be
S_1\equiv S_{n_1}+S_h\,,
\quad
S_2\equiv S_{n_2}+S_h\,,
\quad
S_h\,,
\ee
for the case of $N$ samples of a white GWB in uncorrelated
white detector noise, for a pair of colocated and coaligned 
detectors.
Recall that the likelihood function is
%
\be
p(d|S_{n_1}, S_{n_2}, S_h) =\frac{1}{\sqrt{{\rm det}(2\pi C)}}
\exp\left[-\frac{1}{2}d^T C^{-1} d\right]\,,
\ee
%
where
\be
C
= \left[
\begin{array}{cc}
(S_{n_1} +S_h)\,\unit_{N\times N} & S_h\,\unit_{N\times N}
\\
S_h\,\unit_{N\times N} & (S_{n_2} +S_h)\,\unit_{N\times N}
\\
\end{array}
\right]
\label{e:C_marginalized}
\ee
%
and 
%
\be
d^T C^{-1} d
\equiv \sum_{I,J=1}^2\sum_{i,j=1}^N
d_{Ii} \left(C^{-1}\right)_{Ii,Jj} d_{Jj}\,.
\label{e:argexp}
\ee

\i {\em Derivation of the maximum-likelihood ratio detection statistic:}

Verify that twice the log of the maximum-likelihood
ratio for the standard stochastic likelihood function
goes like the square of the (power) signal-to-noise ratio,
\be
2\ln \Lambda_{\rm ML}(d) \simeq
\frac{\hat C_{12}^2}{\hat C_{11}\hat C_{22}/N}\,,
\ee
in the weak-signal approximation.
({\em Hint:} For simplicity, do the calculation in the context 
of $N$ samples of a white GWB in uncorrelated 
white detector noise, for a pair of colocated and coaligned
detectors, using the results of Exercise~\ref{exer:MLestimators}.)

\i {\em Standard cross-correlation likelihood by marginalizing over 
stochastic signal prior:}

Derive the standard form of the likelihood function
for stochastic background searches 
\be
p(d|S_{n_1}, S_{n_2}, S_h)
=\frac{1}{\sqrt{{\rm det}(2\pi C)}}
\exp\left[-\frac{1}{2} \sum_{I,J=1}^2 d_I \left(C^{-1}\right)_{IJ} d_J\right]\,,
\ee
%
where
%
\be
C\equiv \left[
\begin{array}{cc}
S_{n_1}+S_h & S_h\\
S_h & S_{n_2} + S_h\\
\end{array}
\right]\,,
\ee
by marginalizing 
\be
p_n(d- h|S_{n_1},S_{n_2}) =
\frac{1}{2\pi\sqrt{S_{n_1}S_{n_2}}}
\exp\left[-\frac{1}{2}\left\{
\frac{(d_1- h)^2}{S_{n_1}} + \frac{(d_2- h)^2}{S_{n_2}}
\right\}\right]
\ee
over the signal samples $h$ for the {\em stochastic} signal prior 
\be
p(h|S_h) = \frac{1}{\sqrt{2\pi S_h}}\exp\left[
-\frac{1}{2}\frac{h^2}{S_h}\right]\,.
\ee
%
In other words, show that
%
\be
p(d|S_{n_1}, S_{n_2}, S_h) 
=\int_{-\infty}^\infty \D h\>
p_n(d-h|S_{n_1}, S_{n_2}) p(h|S_h)\,.
\ee
%
({\em Hint}: You'll have to complete the square in the argument
of the exponential in the marginalization integral.)

\een

\newpage
\input refs

\end{document}

